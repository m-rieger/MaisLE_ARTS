% Options for packages loaded elsewhere
\PassOptionsToPackage{unicode}{hyperref}
\PassOptionsToPackage{hyphens}{url}
\PassOptionsToPackage{dvipsnames,svgnames,x11names}{xcolor}
%
\documentclass[
  letterpaper,
  DIV=11,
  numbers=noendperiod]{scrartcl}

\usepackage{amsmath,amssymb}
\usepackage{lmodern}
\usepackage{iftex}
\ifPDFTeX
  \usepackage[T1]{fontenc}
  \usepackage[utf8]{inputenc}
  \usepackage{textcomp} % provide euro and other symbols
\else % if luatex or xetex
  \usepackage{unicode-math}
  \defaultfontfeatures{Scale=MatchLowercase}
  \defaultfontfeatures[\rmfamily]{Ligatures=TeX,Scale=1}
\fi
% Use upquote if available, for straight quotes in verbatim environments
\IfFileExists{upquote.sty}{\usepackage{upquote}}{}
\IfFileExists{microtype.sty}{% use microtype if available
  \usepackage[]{microtype}
  \UseMicrotypeSet[protrusion]{basicmath} % disable protrusion for tt fonts
}{}
\makeatletter
\@ifundefined{KOMAClassName}{% if non-KOMA class
  \IfFileExists{parskip.sty}{%
    \usepackage{parskip}
  }{% else
    \setlength{\parindent}{0pt}
    \setlength{\parskip}{6pt plus 2pt minus 1pt}}
}{% if KOMA class
  \KOMAoptions{parskip=half}}
\makeatother
\usepackage{xcolor}
\setlength{\emergencystretch}{3em} % prevent overfull lines
\setcounter{secnumdepth}{-\maxdimen} % remove section numbering
% Make \paragraph and \subparagraph free-standing
\ifx\paragraph\undefined\else
  \let\oldparagraph\paragraph
  \renewcommand{\paragraph}[1]{\oldparagraph{#1}\mbox{}}
\fi
\ifx\subparagraph\undefined\else
  \let\oldsubparagraph\subparagraph
  \renewcommand{\subparagraph}[1]{\oldsubparagraph{#1}\mbox{}}
\fi


\providecommand{\tightlist}{%
  \setlength{\itemsep}{0pt}\setlength{\parskip}{0pt}}\usepackage{longtable,booktabs,array}
\usepackage{calc} % for calculating minipage widths
% Correct order of tables after \paragraph or \subparagraph
\usepackage{etoolbox}
\makeatletter
\patchcmd\longtable{\par}{\if@noskipsec\mbox{}\fi\par}{}{}
\makeatother
% Allow footnotes in longtable head/foot
\IfFileExists{footnotehyper.sty}{\usepackage{footnotehyper}}{\usepackage{footnote}}
\makesavenoteenv{longtable}
\usepackage{graphicx}
\makeatletter
\def\maxwidth{\ifdim\Gin@nat@width>\linewidth\linewidth\else\Gin@nat@width\fi}
\def\maxheight{\ifdim\Gin@nat@height>\textheight\textheight\else\Gin@nat@height\fi}
\makeatother
% Scale images if necessary, so that they will not overflow the page
% margins by default, and it is still possible to overwrite the defaults
% using explicit options in \includegraphics[width, height, ...]{}
\setkeys{Gin}{width=\maxwidth,height=\maxheight,keepaspectratio}
% Set default figure placement to htbp
\makeatletter
\def\fps@figure{htbp}
\makeatother

\KOMAoption{captions}{tableheading}
\makeatletter
\makeatother
\makeatletter
\makeatother
\makeatletter
\@ifpackageloaded{caption}{}{\usepackage{caption}}
\AtBeginDocument{%
\ifdefined\contentsname
  \renewcommand*\contentsname{Table of contents}
\else
  \newcommand\contentsname{Table of contents}
\fi
\ifdefined\listfigurename
  \renewcommand*\listfigurename{List of Figures}
\else
  \newcommand\listfigurename{List of Figures}
\fi
\ifdefined\listtablename
  \renewcommand*\listtablename{List of Tables}
\else
  \newcommand\listtablename{List of Tables}
\fi
\ifdefined\figurename
  \renewcommand*\figurename{Figure}
\else
  \newcommand\figurename{Figure}
\fi
\ifdefined\tablename
  \renewcommand*\tablename{Table}
\else
  \newcommand\tablename{Table}
\fi
}
\@ifpackageloaded{float}{}{\usepackage{float}}
\floatstyle{ruled}
\@ifundefined{c@chapter}{\newfloat{codelisting}{h}{lop}}{\newfloat{codelisting}{h}{lop}[chapter]}
\floatname{codelisting}{Listing}
\newcommand*\listoflistings{\listof{codelisting}{List of Listings}}
\makeatother
\makeatletter
\@ifpackageloaded{caption}{}{\usepackage{caption}}
\@ifpackageloaded{subcaption}{}{\usepackage{subcaption}}
\makeatother
\makeatletter
\@ifpackageloaded{tcolorbox}{}{\usepackage[many]{tcolorbox}}
\makeatother
\makeatletter
\@ifundefined{shadecolor}{\definecolor{shadecolor}{rgb}{.97, .97, .97}}
\makeatother
\makeatletter
\makeatother
\ifLuaTeX
  \usepackage{selnolig}  % disable illegal ligatures
\fi
\IfFileExists{bookmark.sty}{\usepackage{bookmark}}{\usepackage{hyperref}}
\IfFileExists{xurl.sty}{\usepackage{xurl}}{} % add URL line breaks if available
\urlstyle{same} % disable monospaced font for URLs
\hypersetup{
  pdftitle={ARTS - methods},
  pdfauthor={Mirjam R. Rieger, Jan Schmitt, Jannis Gottwald, Jonas Hoechst, Patrick Lampe, Paula Machin, Johann Musculus, Ralf Dittrich},
  colorlinks=true,
  linkcolor={blue},
  filecolor={Maroon},
  citecolor={Blue},
  urlcolor={Blue},
  pdfcreator={LaTeX via pandoc}}

\title{ARTS - methods}
\author{Mirjam R. Rieger, Jan Schmitt, Jannis Gottwald, Jonas Hoechst,
Patrick Lampe, Paula Machin, Johann Musculus, Ralf Dittrich}
\date{}

\begin{document}
\maketitle
\ifdefined\Shaded\renewenvironment{Shaded}{\begin{tcolorbox}[boxrule=0pt, sharp corners, breakable, enhanced, frame hidden, interior hidden, borderline west={3pt}{0pt}{shadecolor}]}{\end{tcolorbox}}\fi

\hypertarget{about}{%
\section*{About}\label{about}}
\addcontentsline{toc}{section}{About}

Paper to compare different methods for position estimation using
directional stations (quadrologgers) and omnidiretional stations
(monologgers), mainly based on data from maisC but supplemented with
data from maisD and melons (only quadrologgers).

We aim to publish it in Methods of Ecology and Evolution
{[}https://besjournals.onlinelibrary.wiley.com/journal/2041210x{]}

\hypertarget{abstract}{%
\section*{Abstract}\label{abstract}}
\addcontentsline{toc}{section}{Abstract}

\hypertarget{introduction}{%
\section{Introduction}\label{introduction}}

\hypertarget{methods}{%
\section{Methods}\label{methods}}

\hypertarget{stations}{%
\subsection{stations}\label{stations}}

\hypertarget{quadrologgers}{%
\subsubsection{quadrologgers}\label{quadrologgers}}

A station with 4 directional antennas, usually facing north, east,
south, and west. Antennas used were \textbf{Yagi} antennas (decribe
model and type, e.g.~refer to already existing paper with same setup)\\
\emph{--\textgreater{} show picture of antenna beams (by Ralf)}

\hypertarget{monologgers}{%
\subsubsection{monologgers}\label{monologgers}}

A station with 1 omnidirectional antenna facing upwards. Antennas used
were \textbf{????} antennas (describe model and type, e.g.~refer to
already existing paper with same setup)\\
\emph{--\textgreater{} show picture of antenna beam (tRackIT??)}

\hypertarget{tags}{%
\subsection{tags}\label{tags}}

\emph{--\textgreater{} describe different types of testtags (for maisC
and D same models, melons has different ones)}

\hypertarget{sample-sites}{%
\subsection{sample sites}\label{sample-sites}}

\emph{--\textgreater{} maybe make a table to compare the different sites
with columns no. of quadrologgers, no. of monologgers, no. and type of
testtags, no. of testtracks, no. of circle tracks, no. of grid points
(and distance between them), site descirption (e.g.~difference in
elevation, vegetation, \ldots)}

\hypertarget{maisc}{%
\subsubsection{maisC}\label{maisc}}

\begin{itemize}
\tightlist
\item
  \textbf{10 quadrologgers} (maybe only 7 will be used due to twisted
  stations)\\
\item
  \textbf{10 monologgers} (maybe only 7 corresponding stations will be
  used)
\end{itemize}

3-4 testtags at different heights (0.5m, 1m, 1.5m, 2m)\\
- \textbf{XX testtracks}\\
- \textbf{circle tracks} for stations c1l1-c6l1 (50m, 100m, 150m)\\
- \textbf{gridpoints} with 100m distance to each other within a 300m (or
400m???) radius around all stations

\hypertarget{maisd}{%
\subsubsection{maisD}\label{maisd}}

\begin{itemize}
\tightlist
\item
  \textbf{8 quadrologgers}
\end{itemize}

3 testtags at different heights (0.5m, 1m, 1.5m)\\
- \textbf{XX testtracks}\\
- \textbf{circle tracks} for all stations (50m, 100m, 150m)\\
- \textbf{gridpoints} with 100m distance to each other within a 300m (or
400m???) radius around all stations

\hypertarget{melons}{%
\subsubsection{melons}\label{melons}}

\begin{itemize}
\tightlist
\item
  \textbf{xx quadrologgers} (ask Paula)
\end{itemize}

xx testtags at different heights (???)\\
- \textbf{XX testtracks}\\
- \textbf{circle tracks} for xx stations (50m, 100m, 150m)\\
- \textbf{gridpoints} with ???m distance to each other within a 300m (or
400m???) radius around all stations

\hypertarget{preparation-of-data}{%
\subsection{preparation of data}\label{preparation-of-data}}

\emph{--\textgreater{} explain filtering process done by tRackIT}

\hypertarget{position-estimation}{%
\subsection{position estimation}\label{position-estimation}}

We used xx different approaches for positions estimations, namely\\
- \textbf{bearings and triangulations}, e.g.~method xx and method xy\\
- \textbf{antenna beams} for quadrologgers\\
- \textbf{multilateration} (e.g.~another name) for monologgers\\
- \textbf{mix of both}\\
- \ldots{}

\emph{--\textgreater{} make a table as overview (e.g.~which method is
used for which type of station, \ldots)}

\hypertarget{comparison-of-methods}{%
\subsection{comparison of methods}\label{comparison-of-methods}}

\hypertarget{bearing-accuracy}{%
\subsubsection{bearing accuracy}\label{bearing-accuracy}}

For bearing and triangulation methods only. Uses deviance between true
angle (based on GPS position) and estimated angle. Should be accumulated
around 0. If it is constantly shifted to the left or right, this might
be an indicator that the antenna itself was shifted by some degrees.
This might be used as a correction factor (also for other approaches) to
redefine the northern orientation. If it shifts over time, one might
either consider using different correction factors for the northern
orientation over time or to exclude the station



\end{document}
